\documentclass{article}
\usepackage{arxiv}

\usepackage[utf8]{inputenc}
\usepackage[english, russian]{babel}
\usepackage[T1]{fontenc}
\usepackage{url}
\usepackage{booktabs}
\usepackage{amsfonts}
\usepackage{nicefrac}
\usepackage{microtype}
\usepackage{lipsum}
\usepackage{graphicx}
\usepackage{natbib}
\usepackage{doi}



\title{Декодирования сигналов головного мозга в аудиоданные}

\author{ Набиев Мухаммадшариф \\
        Кафедра интеллектуальных систем\\
	МФТИ\\
	\texttt{nabiev.mf@phystech} \\
	\And
	Северилов Павел \\
	Кафедра интеллектульных систем\\
	МФТИ\\
	\texttt{pseverilov@gmail.com} \\
	%% \AND
	%% Coauthor \\
	%% Affiliation \\
	%% Address \\
	%% \texttt{email} \\
	%% \And
	%% Coauthor \\
	%% Affiliation \\
	%% Address \\
	%% \texttt{email} \\
	%% \And
	%% Coauthor \\
	%% Affiliation \\
	%% Address \\
	%% \texttt{email} \\
}
\date{}

\renewcommand{\shorttitle}{\textit{arXiv} Template}

%%% Add PDF metadata to help others organize their library
%%% Once the PDF is generated, you can check the metadata with
%%% $ pdfinfo template.pdf
\hypersetup{
pdftitle={Декодирования сигналов головного мозга в аудиоданные},
pdfsubject={q-bio.NC, q-bio.QM},
pdfauthor={Набиев Мухаммадшариф, Северилов Павел},
pdfkeywords={First keyword, Second keyword, More},
}

\begin{document}
\maketitle

\begin{abstract}
    В данной статье исследуется проблема декодирования сигналов головного мозга в аудиоданные с использованием современных методов получения эмбеддингов аудиоданных. Мы обращаемся к методу Wav2vec для извлечения высококачественных представлений аудиоинформации из сигналов электроэнцефалограммы (ЭЭГ). Эти методы позволяют нам не только решать задачи классификации и регрессии в контексте аудиоданных, но и представляют новые возможности для улучшения качества декодирования сигналов мозга. Полученные результаты имеют важное значение для развития интерфейсов мозг-компьютер и глубокого понимания принципов аудиальной обработки информации человеческим мозгом. Для достижения этой цели используется датасет, собранный Левёнским университетом. Датасет состоит из 668 пар ЭЭГ-стимул общей продолжительностью 9431 минута. 
 
\end{abstract}


\keywords{auditory EEG decoding \and natural speech processing \and EEG recordings}

\section{Введение}



\bibliographystyle{unsrtnat}
\bibliography{references}

\end{document}